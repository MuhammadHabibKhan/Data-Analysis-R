% Options for packages loaded elsewhere
\PassOptionsToPackage{unicode}{hyperref}
\PassOptionsToPackage{hyphens}{url}
%
\documentclass[
]{article}
\usepackage{amsmath,amssymb}
\usepackage{lmodern}
\usepackage{ifxetex,ifluatex}
\ifnum 0\ifxetex 1\fi\ifluatex 1\fi=0 % if pdftex
  \usepackage[T1]{fontenc}
  \usepackage[utf8]{inputenc}
  \usepackage{textcomp} % provide euro and other symbols
\else % if luatex or xetex
  \usepackage{unicode-math}
  \defaultfontfeatures{Scale=MatchLowercase}
  \defaultfontfeatures[\rmfamily]{Ligatures=TeX,Scale=1}
\fi
% Use upquote if available, for straight quotes in verbatim environments
\IfFileExists{upquote.sty}{\usepackage{upquote}}{}
\IfFileExists{microtype.sty}{% use microtype if available
  \usepackage[]{microtype}
  \UseMicrotypeSet[protrusion]{basicmath} % disable protrusion for tt fonts
}{}
\makeatletter
\@ifundefined{KOMAClassName}{% if non-KOMA class
  \IfFileExists{parskip.sty}{%
    \usepackage{parskip}
  }{% else
    \setlength{\parindent}{0pt}
    \setlength{\parskip}{6pt plus 2pt minus 1pt}}
}{% if KOMA class
  \KOMAoptions{parskip=half}}
\makeatother
\usepackage{xcolor}
\IfFileExists{xurl.sty}{\usepackage{xurl}}{} % add URL line breaks if available
\IfFileExists{bookmark.sty}{\usepackage{bookmark}}{\usepackage{hyperref}}
\hypersetup{
  pdftitle={B20102088-MuhammadHabibKhan-FinalLab},
  pdfauthor={Muhammad Habib Khan},
  hidelinks,
  pdfcreator={LaTeX via pandoc}}
\urlstyle{same} % disable monospaced font for URLs
\usepackage[margin=1in]{geometry}
\usepackage{color}
\usepackage{fancyvrb}
\newcommand{\VerbBar}{|}
\newcommand{\VERB}{\Verb[commandchars=\\\{\}]}
\DefineVerbatimEnvironment{Highlighting}{Verbatim}{commandchars=\\\{\}}
% Add ',fontsize=\small' for more characters per line
\usepackage{framed}
\definecolor{shadecolor}{RGB}{248,248,248}
\newenvironment{Shaded}{\begin{snugshade}}{\end{snugshade}}
\newcommand{\AlertTok}[1]{\textcolor[rgb]{0.94,0.16,0.16}{#1}}
\newcommand{\AnnotationTok}[1]{\textcolor[rgb]{0.56,0.35,0.01}{\textbf{\textit{#1}}}}
\newcommand{\AttributeTok}[1]{\textcolor[rgb]{0.77,0.63,0.00}{#1}}
\newcommand{\BaseNTok}[1]{\textcolor[rgb]{0.00,0.00,0.81}{#1}}
\newcommand{\BuiltInTok}[1]{#1}
\newcommand{\CharTok}[1]{\textcolor[rgb]{0.31,0.60,0.02}{#1}}
\newcommand{\CommentTok}[1]{\textcolor[rgb]{0.56,0.35,0.01}{\textit{#1}}}
\newcommand{\CommentVarTok}[1]{\textcolor[rgb]{0.56,0.35,0.01}{\textbf{\textit{#1}}}}
\newcommand{\ConstantTok}[1]{\textcolor[rgb]{0.00,0.00,0.00}{#1}}
\newcommand{\ControlFlowTok}[1]{\textcolor[rgb]{0.13,0.29,0.53}{\textbf{#1}}}
\newcommand{\DataTypeTok}[1]{\textcolor[rgb]{0.13,0.29,0.53}{#1}}
\newcommand{\DecValTok}[1]{\textcolor[rgb]{0.00,0.00,0.81}{#1}}
\newcommand{\DocumentationTok}[1]{\textcolor[rgb]{0.56,0.35,0.01}{\textbf{\textit{#1}}}}
\newcommand{\ErrorTok}[1]{\textcolor[rgb]{0.64,0.00,0.00}{\textbf{#1}}}
\newcommand{\ExtensionTok}[1]{#1}
\newcommand{\FloatTok}[1]{\textcolor[rgb]{0.00,0.00,0.81}{#1}}
\newcommand{\FunctionTok}[1]{\textcolor[rgb]{0.00,0.00,0.00}{#1}}
\newcommand{\ImportTok}[1]{#1}
\newcommand{\InformationTok}[1]{\textcolor[rgb]{0.56,0.35,0.01}{\textbf{\textit{#1}}}}
\newcommand{\KeywordTok}[1]{\textcolor[rgb]{0.13,0.29,0.53}{\textbf{#1}}}
\newcommand{\NormalTok}[1]{#1}
\newcommand{\OperatorTok}[1]{\textcolor[rgb]{0.81,0.36,0.00}{\textbf{#1}}}
\newcommand{\OtherTok}[1]{\textcolor[rgb]{0.56,0.35,0.01}{#1}}
\newcommand{\PreprocessorTok}[1]{\textcolor[rgb]{0.56,0.35,0.01}{\textit{#1}}}
\newcommand{\RegionMarkerTok}[1]{#1}
\newcommand{\SpecialCharTok}[1]{\textcolor[rgb]{0.00,0.00,0.00}{#1}}
\newcommand{\SpecialStringTok}[1]{\textcolor[rgb]{0.31,0.60,0.02}{#1}}
\newcommand{\StringTok}[1]{\textcolor[rgb]{0.31,0.60,0.02}{#1}}
\newcommand{\VariableTok}[1]{\textcolor[rgb]{0.00,0.00,0.00}{#1}}
\newcommand{\VerbatimStringTok}[1]{\textcolor[rgb]{0.31,0.60,0.02}{#1}}
\newcommand{\WarningTok}[1]{\textcolor[rgb]{0.56,0.35,0.01}{\textbf{\textit{#1}}}}
\usepackage{longtable,booktabs,array}
\usepackage{calc} % for calculating minipage widths
% Correct order of tables after \paragraph or \subparagraph
\usepackage{etoolbox}
\makeatletter
\patchcmd\longtable{\par}{\if@noskipsec\mbox{}\fi\par}{}{}
\makeatother
% Allow footnotes in longtable head/foot
\IfFileExists{footnotehyper.sty}{\usepackage{footnotehyper}}{\usepackage{footnote}}
\makesavenoteenv{longtable}
\usepackage{graphicx}
\makeatletter
\def\maxwidth{\ifdim\Gin@nat@width>\linewidth\linewidth\else\Gin@nat@width\fi}
\def\maxheight{\ifdim\Gin@nat@height>\textheight\textheight\else\Gin@nat@height\fi}
\makeatother
% Scale images if necessary, so that they will not overflow the page
% margins by default, and it is still possible to overwrite the defaults
% using explicit options in \includegraphics[width, height, ...]{}
\setkeys{Gin}{width=\maxwidth,height=\maxheight,keepaspectratio}
% Set default figure placement to htbp
\makeatletter
\def\fps@figure{htbp}
\makeatother
\setlength{\emergencystretch}{3em} % prevent overfull lines
\providecommand{\tightlist}{%
  \setlength{\itemsep}{0pt}\setlength{\parskip}{0pt}}
\setcounter{secnumdepth}{-\maxdimen} % remove section numbering
\ifluatex
  \usepackage{selnolig}  % disable illegal ligatures
\fi

\title{B20102088-MuhammadHabibKhan-FinalLab}
\author{Muhammad Habib Khan}
\date{1/22/2022}

\begin{document}
\maketitle

{
\setcounter{tocdepth}{2}
\tableofcontents
}
\begin{Shaded}
\begin{Highlighting}[]
\CommentTok{\# Converting Time Series into Data Frame Structure}

\FunctionTok{data}\NormalTok{(Seatbelts)}
\NormalTok{Seatbelts }\OtherTok{\textless{}{-}} \FunctionTok{data.frame}\NormalTok{(}\AttributeTok{Year=}\FunctionTok{floor}\NormalTok{(}\FunctionTok{time}\NormalTok{(Seatbelts)),}
\AttributeTok{Month=}\FunctionTok{factor}\NormalTok{(}\FunctionTok{cycle}\NormalTok{(Seatbelts),}
\AttributeTok{labels=}\NormalTok{month.abb), Seatbelts)}

\NormalTok{knitr}\SpecialCharTok{::}\NormalTok{opts\_chunk}\SpecialCharTok{$}\FunctionTok{set}\NormalTok{(}\AttributeTok{echo =} \ConstantTok{TRUE}\NormalTok{)}

\FunctionTok{library}\NormalTok{(knitr)}
\end{Highlighting}
\end{Shaded}

\hypertarget{selecting-a-dataset}{%
\section{Selecting a Dataset}\label{selecting-a-dataset}}

Data Set Name: \emph{Seatbelts}

Data Set From built-in R Data sets in package `datasets'

\textbf{Abstract}: Road Casualties in Great Britain 1969-84

\textbf{Data Set Information:}

The `Seatbelts' data set in R is a multiple time-series data set that
was commissioned by the Department of Transport in 1984 to measure
differences in deaths before and after front seat belt legislation was
introduced on 31st January 1983. It provides monthly total numerical
data on a number of incidents including those related to death and
injury in Road Traffic Accidents (RTA's). The data set starts in January
1969 and observations run until December 1984.

\textbf{Attributes:}

DriversKilled: Car drivers killed.

drivers: Same as \emph{UKDriverDeaths} deaths count.

\emph{UKDriverDeaths is a time series giving the monthly totals of car
drivers in Great Britain killed or seriously injured Jan 1969 to Dec
1984. Compulsory wearing of seat belts was introduced on 31 Jan 1983.}
\emph{Seat belts is more information on the same problem.}

front: Front-seat passengers killed or seriously injured.

rear: Rear-seat passengers killed or seriously injured.

kms: Distance driven.

PetrolPrice: Petrol price.

VanKilled: Number of van (`light goods vehicle') drivers killed.

law: 0/1: Was the law in effect that month?

\hypertarget{data-set-at-a-glance}{%
\section{Data Set At A Glance}\label{data-set-at-a-glance}}

\begin{Shaded}
\begin{Highlighting}[]
\CommentTok{\# Summary of the Entire Data}
\FunctionTok{summary}\NormalTok{(Seatbelts)}
\end{Highlighting}
\end{Shaded}

\begin{verbatim}
##       Year          Month    DriversKilled      drivers         front       
##  Min.   :1969   Jan    :16   Min.   : 60.0   Min.   :1057   Min.   : 426.0  
##  1st Qu.:1973   Feb    :16   1st Qu.:104.8   1st Qu.:1462   1st Qu.: 715.5  
##  Median :1976   Mar    :16   Median :118.5   Median :1631   Median : 828.5  
##  Mean   :1976   Apr    :16   Mean   :122.8   Mean   :1670   Mean   : 837.2  
##  3rd Qu.:1980   May    :16   3rd Qu.:138.0   3rd Qu.:1851   3rd Qu.: 950.8  
##  Max.   :1984   Jun    :16   Max.   :198.0   Max.   :2654   Max.   :1299.0  
##                 (Other):96                                                  
##       rear            kms         PetrolPrice        VanKilled     
##  Min.   :224.0   Min.   : 7685   Min.   :0.08118   Min.   : 2.000  
##  1st Qu.:344.8   1st Qu.:12685   1st Qu.:0.09258   1st Qu.: 6.000  
##  Median :401.5   Median :14987   Median :0.10448   Median : 8.000  
##  Mean   :401.2   Mean   :14994   Mean   :0.10362   Mean   : 9.057  
##  3rd Qu.:456.2   3rd Qu.:17202   3rd Qu.:0.11406   3rd Qu.:12.000  
##  Max.   :646.0   Max.   :21626   Max.   :0.13303   Max.   :17.000  
##                                                                    
##       law        
##  Min.   :0.0000  
##  1st Qu.:0.0000  
##  Median :0.0000  
##  Mean   :0.1198  
##  3rd Qu.:0.0000  
##  Max.   :1.0000  
## 
\end{verbatim}

\begin{Shaded}
\begin{Highlighting}[]
\CommentTok{\# Gist of the Data including only few of the rows}
\FunctionTok{head}\NormalTok{(Seatbelts, }\DecValTok{5}\NormalTok{)}
\end{Highlighting}
\end{Shaded}

\begin{verbatim}
##   Year Month DriversKilled drivers front rear   kms PetrolPrice VanKilled law
## 1 1969   Jan           107    1687   867  269  9059   0.1029718        12   0
## 2 1969   Feb            97    1508   825  265  7685   0.1023630         6   0
## 3 1969   Mar           102    1507   806  319  9963   0.1020625        12   0
## 4 1969   Apr            87    1385   814  407 10955   0.1008733         8   0
## 5 1969   May           119    1632   991  454 11823   0.1010197        10   0
\end{verbatim}

\begin{Shaded}
\begin{Highlighting}[]
\CommentTok{\# Another useful command to look at the summarized data and skim it}

\FunctionTok{library}\NormalTok{(skimr)}
\FunctionTok{skim}\NormalTok{(Seatbelts)}
\end{Highlighting}
\end{Shaded}

\begin{longtable}[]{@{}ll@{}}
\caption{Data summary}\tabularnewline
\toprule
& \\
\midrule
\endfirsthead
\toprule
& \\
\midrule
\endhead
Name & Seatbelts \\
Number of rows & 192 \\
Number of columns & 10 \\
\_\_\_\_\_\_\_\_\_\_\_\_\_\_\_\_\_\_\_\_\_\_\_ & \\
Column type frequency: & \\
factor & 1 \\
numeric & 8 \\
ts & 1 \\
\_\_\_\_\_\_\_\_\_\_\_\_\_\_\_\_\_\_\_\_\_\_\_\_ & \\
Group variables & None \\
\bottomrule
\end{longtable}

\textbf{Variable type: factor}

\begin{longtable}[]{@{}lrrlrl@{}}
\toprule
skim\_variable & n\_missing & complete\_rate & ordered & n\_unique &
top\_counts \\
\midrule
\endhead
Month & 0 & 1 & FALSE & 12 & Jan: 16, Feb: 16, Mar: 16, Apr: 16 \\
\bottomrule
\end{longtable}

\textbf{Variable type: numeric}

\begin{longtable}[]{@{}lrrrrrrrrrl@{}}
\toprule
skim\_variable & n\_missing & complete\_rate & mean & sd & p0 & p25 &
p50 & p75 & p100 & hist \\
\midrule
\endhead
DriversKilled & 0 & 1 & 122.80 & 25.38 & 60.00 & 104.75 & 118.5 & 138.00
& 198.00 & ▁▇▆▃▁ \\
drivers & 0 & 1 & 1670.31 & 289.61 & 1057.00 & 1461.75 & 1631.0 &
1850.75 & 2654.00 & ▂▇▃▂▁ \\
front & 0 & 1 & 837.22 & 175.10 & 426.00 & 715.50 & 828.5 & 950.75 &
1299.00 & ▂▆▇▅▁ \\
rear & 0 & 1 & 401.21 & 83.10 & 224.00 & 344.75 & 401.5 & 456.25 &
646.00 & ▃▇▇▂▁ \\
kms & 0 & 1 & 14993.60 & 2938.05 & 7685.00 & 12685.00 & 14987.0 &
17202.50 & 21626.00 & ▂▆▇▇▂ \\
PetrolPrice & 0 & 1 & 0.10 & 0.01 & 0.08 & 0.09 & 0.1 & 0.11 & 0.13 &
▆▅▇▇▁ \\
VanKilled & 0 & 1 & 9.06 & 3.64 & 2.00 & 6.00 & 8.0 & 12.00 & 17.00 &
▅▇▅▆▂ \\
law & 0 & 1 & 0.12 & 0.33 & 0.00 & 0.00 & 0.0 & 0.00 & 1.00 & ▇▁▁▁▁ \\
\bottomrule
\end{longtable}

\textbf{Variable type: ts}

\begin{longtable}[]{@{}
  >{\raggedright\arraybackslash}p{(\columnwidth - 24\tabcolsep) * \real{0.13}}
  >{\raggedleft\arraybackslash}p{(\columnwidth - 24\tabcolsep) * \real{0.09}}
  >{\raggedleft\arraybackslash}p{(\columnwidth - 24\tabcolsep) * \real{0.13}}
  >{\raggedleft\arraybackslash}p{(\columnwidth - 24\tabcolsep) * \real{0.06}}
  >{\raggedleft\arraybackslash}p{(\columnwidth - 24\tabcolsep) * \real{0.05}}
  >{\raggedleft\arraybackslash}p{(\columnwidth - 24\tabcolsep) * \real{0.09}}
  >{\raggedleft\arraybackslash}p{(\columnwidth - 24\tabcolsep) * \real{0.07}}
  >{\raggedleft\arraybackslash}p{(\columnwidth - 24\tabcolsep) * \real{0.07}}
  >{\raggedleft\arraybackslash}p{(\columnwidth - 24\tabcolsep) * \real{0.05}}
  >{\raggedleft\arraybackslash}p{(\columnwidth - 24\tabcolsep) * \real{0.05}}
  >{\raggedleft\arraybackslash}p{(\columnwidth - 24\tabcolsep) * \real{0.05}}
  >{\raggedleft\arraybackslash}p{(\columnwidth - 24\tabcolsep) * \real{0.07}}
  >{\raggedright\arraybackslash}p{(\columnwidth - 24\tabcolsep) * \real{0.10}}@{}}
\toprule
skim\_variable & n\_missing & complete\_rate & start & end & frequency &
deltat & mean & sd & min & max & median & line\_graph \\
\midrule
\endhead
Year & 0 & 1 & 1969 & 1984 & 12 & 0.08 & 1976.5 & 4.62 & 1969 & 1984 &
1976.5 & ⣀⣀⠤⠤⠒⠒⠉⠉ \\
\bottomrule
\end{longtable}

\begin{Shaded}
\begin{Highlighting}[]
\CommentTok{\# Another command to displays the type and a preview of all columns as a row }
\CommentTok{\# so that it\textquotesingle{}s very easy to take in and easily preview data type and sample data.}

\FunctionTok{library}\NormalTok{(dplyr)}
\end{Highlighting}
\end{Shaded}

\begin{verbatim}
## 
## Attaching package: 'dplyr'
\end{verbatim}

\begin{verbatim}
## The following objects are masked from 'package:stats':
## 
##     filter, lag
\end{verbatim}

\begin{verbatim}
## The following objects are masked from 'package:base':
## 
##     intersect, setdiff, setequal, union
\end{verbatim}

\begin{Shaded}
\begin{Highlighting}[]
\FunctionTok{glimpse}\NormalTok{(Seatbelts)}
\end{Highlighting}
\end{Shaded}

\begin{verbatim}
## Rows: 192
## Columns: 10
## $ Year          <dbl> 1969, 1969, 1969, 1969, 1969, 1969, 1969, 1969, 1969, 19~
## $ Month         <fct> Jan, Feb, Mar, Apr, May, Jun, Jul, Aug, Sep, Oct, Nov, D~
## $ DriversKilled <dbl> 107, 97, 102, 87, 119, 106, 110, 106, 107, 134, 147, 180~
## $ drivers       <dbl> 1687, 1508, 1507, 1385, 1632, 1511, 1559, 1630, 1579, 16~
## $ front         <dbl> 867, 825, 806, 814, 991, 945, 1004, 1091, 958, 850, 1109~
## $ rear          <dbl> 269, 265, 319, 407, 454, 427, 522, 536, 405, 437, 434, 4~
## $ kms           <dbl> 9059, 7685, 9963, 10955, 11823, 12391, 13460, 14055, 121~
## $ PetrolPrice   <dbl> 0.10297181, 0.10236300, 0.10206249, 0.10087330, 0.101019~
## $ VanKilled     <dbl> 12, 6, 12, 8, 10, 13, 11, 6, 10, 16, 13, 14, 14, 6, 8, 1~
## $ law           <dbl> 0, 0, 0, 0, 0, 0, 0, 0, 0, 0, 0, 0, 0, 0, 0, 0, 0, 0, 0,~
\end{verbatim}

\begin{Shaded}
\begin{Highlighting}[]
\CommentTok{\# A comprehensive look at the entire data set produced in html format }
\CommentTok{\# It includes a ton of information including the basic statistics, structure, }
\CommentTok{\# missing data, distribution visualizations, correlation matrix and principal }
\CommentTok{\# component analysis. The screenshots of the webpage are attached below.}

\FunctionTok{library}\NormalTok{(DataExplorer)}
\NormalTok{DataExplorer}\SpecialCharTok{::}\FunctionTok{create\_report}\NormalTok{(datasets}\SpecialCharTok{::}\NormalTok{Seatbelts)}
\end{Highlighting}
\end{Shaded}

\begin{verbatim}
## 
## 
## processing file: report.rmd
\end{verbatim}

\begin{verbatim}
##   |                                                                              |                                                                      |   0%  |                                                                              |..                                                                    |   2%
##    inline R code fragments
## 
##   |                                                                              |...                                                                   |   5%
## label: global_options (with options) 
## List of 1
##  $ include: logi FALSE
## 
##   |                                                                              |.....                                                                 |   7%
##   ordinary text without R code
## 
##   |                                                                              |.......                                                               |  10%
## label: introduce
##   |                                                                              |........                                                              |  12%
##   ordinary text without R code
## 
##   |                                                                              |..........                                                            |  14%
## label: plot_intro
\end{verbatim}

\begin{verbatim}
##   |                                                                              |............                                                          |  17%
##   ordinary text without R code
## 
##   |                                                                              |.............                                                         |  19%
## label: data_structure
##   |                                                                              |...............                                                       |  21%
##   ordinary text without R code
## 
##   |                                                                              |.................                                                     |  24%
## label: missing_profile
\end{verbatim}

\begin{verbatim}
##   |                                                                              |..................                                                    |  26%
##   ordinary text without R code
## 
##   |                                                                              |....................                                                  |  29%
## label: univariate_distribution_header
##   |                                                                              |......................                                                |  31%
##   ordinary text without R code
## 
##   |                                                                              |.......................                                               |  33%
## label: plot_histogram
\end{verbatim}

\begin{verbatim}
##   |                                                                              |.........................                                             |  36%
##   ordinary text without R code
## 
##   |                                                                              |...........................                                           |  38%
## label: plot_density
##   |                                                                              |............................                                          |  40%
##   ordinary text without R code
## 
##   |                                                                              |..............................                                        |  43%
## label: plot_frequency_bar
##   |                                                                              |................................                                      |  45%
##   ordinary text without R code
## 
##   |                                                                              |.................................                                     |  48%
## label: plot_response_bar
##   |                                                                              |...................................                                   |  50%
##   ordinary text without R code
## 
##   |                                                                              |.....................................                                 |  52%
## label: plot_with_bar
##   |                                                                              |......................................                                |  55%
##   ordinary text without R code
## 
##   |                                                                              |........................................                              |  57%
## label: plot_normal_qq
\end{verbatim}

\begin{verbatim}
##   |                                                                              |..........................................                            |  60%
##   ordinary text without R code
## 
##   |                                                                              |...........................................                           |  62%
## label: plot_response_qq
##   |                                                                              |.............................................                         |  64%
##   ordinary text without R code
## 
##   |                                                                              |...............................................                       |  67%
## label: plot_by_qq
##   |                                                                              |................................................                      |  69%
##   ordinary text without R code
## 
##   |                                                                              |..................................................                    |  71%
## label: correlation_analysis
\end{verbatim}

\begin{verbatim}
##   |                                                                              |....................................................                  |  74%
##   ordinary text without R code
## 
##   |                                                                              |.....................................................                 |  76%
## label: principal_component_analysis
\end{verbatim}

\begin{verbatim}
##   |                                                                              |.......................................................               |  79%
##   ordinary text without R code
## 
##   |                                                                              |.........................................................             |  81%
## label: bivariate_distribution_header
##   |                                                                              |..........................................................            |  83%
##   ordinary text without R code
## 
##   |                                                                              |............................................................          |  86%
## label: plot_response_boxplot
##   |                                                                              |..............................................................        |  88%
##   ordinary text without R code
## 
##   |                                                                              |...............................................................       |  90%
## label: plot_by_boxplot
##   |                                                                              |.................................................................     |  93%
##   ordinary text without R code
## 
##   |                                                                              |...................................................................   |  95%
## label: plot_response_scatterplot
##   |                                                                              |....................................................................  |  98%
##   ordinary text without R code
## 
##   |                                                                              |......................................................................| 100%
## label: plot_by_scatterplot
\end{verbatim}

\begin{verbatim}
## output file: /Users/mhk/Desktop/report.knit.md
\end{verbatim}

\begin{verbatim}
## /Applications/RStudio.app/Contents/MacOS/pandoc/pandoc +RTS -K512m -RTS /Users/mhk/Desktop/report.knit.md --to html4 --from markdown+autolink_bare_uris+tex_math_single_backslash --output /Users/mhk/Desktop/report.html --lua-filter /Library/Frameworks/R.framework/Versions/4.1/Resources/library/rmarkdown/rmarkdown/lua/pagebreak.lua --lua-filter /Library/Frameworks/R.framework/Versions/4.1/Resources/library/rmarkdown/rmarkdown/lua/latex-div.lua --self-contained --variable bs3=TRUE --standalone --section-divs --table-of-contents --toc-depth 6 --template /Library/Frameworks/R.framework/Versions/4.1/Resources/library/rmarkdown/rmd/h/default.html --no-highlight --variable highlightjs=1 --variable theme=yeti --include-in-header /var/folders/26/dl06hcqn4n377sbrgs92p4jr0000gn/T//RtmpPgrwOl/rmarkdown-str8c09184f6bfc.html --mathjax --variable 'mathjax-url:https://mathjax.rstudio.com/latest/MathJax.js?config=TeX-AMS-MML_HTMLorMML'
\end{verbatim}

\begin{verbatim}
## 
## Output created: report.html
\end{verbatim}

\hypertarget{basic-plots}{%
\section{Basic Plots}\label{basic-plots}}

Some basic plots, charts and graphs of the data set and its classes are
shown below which includes:

\begin{enumerate}
\def\labelenumi{\arabic{enumi})}
\tightlist
\item
  Bar Plot
\item
  3D Pie Chart
\item
  Histogram
\item
  Scatter Plot
\item
  Box Plot
\end{enumerate}

\hypertarget{bar-plot}{%
\subsection{Bar Plot:}\label{bar-plot}}

\begin{Shaded}
\begin{Highlighting}[]
\FunctionTok{attach}\NormalTok{(Seatbelts)}

\CommentTok{\# Bar{-}plot of Number of Van Drivers killed through out the years}

\FunctionTok{barplot}\NormalTok{(Seatbelts}\SpecialCharTok{$}\NormalTok{VanKilled, }\AttributeTok{xlab=}\StringTok{\textquotesingle{}Time (months)\textquotesingle{}}\NormalTok{, }\AttributeTok{ylab=}\StringTok{\textquotesingle{}Number of Van Drivers Killed\textquotesingle{}}\NormalTok{, }\AttributeTok{col=}\FunctionTok{heat.colors}\NormalTok{(}\DecValTok{10}\NormalTok{))}
\end{Highlighting}
\end{Shaded}

\includegraphics{B20102088-MuhammadHabibKhan-FinalLab_files/figure-latex/unnamed-chunk-2-1.pdf}

The bar plot above shows the number of Van Drivers killed during a car
accident in Britain in the described time period. While the numbers were
small to begin with, we can see the entire pattern going down on a
negative slope as they month passes, indicating that when the law of
seat belts being mandatory was passed, the amount of deaths for van
drivers decreased too even though the time was beginning of Industrial
Revolution and the cars on the roads increased as the days passed.

\hypertarget{d-pie-chart}{%
\subsection{3D Pie Chart:}\label{d-pie-chart}}

\begin{Shaded}
\begin{Highlighting}[]
\CommentTok{\# A 3D pie chart of the mean of the number of car drivers, van drivers, front seat and rear seat passengers killed through out Jan, 1969 {-} Dec, 1984}

\FunctionTok{library}\NormalTok{(plotrix)}
\NormalTok{x }\OtherTok{\textless{}{-}} \FunctionTok{c}\NormalTok{(}\FunctionTok{mean}\NormalTok{(Seatbelts}\SpecialCharTok{$}\NormalTok{DriversKilled), }\FunctionTok{mean}\NormalTok{(Seatbelts}\SpecialCharTok{$}\NormalTok{front), }\FunctionTok{mean}\NormalTok{(Seatbelts}\SpecialCharTok{$}\NormalTok{rear), }\FunctionTok{mean}\NormalTok{(Seatbelts}\SpecialCharTok{$}\NormalTok{VanKilled))}
\FunctionTok{names}\NormalTok{(x) }\OtherTok{\textless{}{-}} \FunctionTok{c}\NormalTok{(}\StringTok{"DriversKilled"}\NormalTok{, }\StringTok{"front"}\NormalTok{, }\StringTok{"rear"}\NormalTok{, }\StringTok{"VanKilled"}\NormalTok{)}
\FunctionTok{pie3D}\NormalTok{(x, }\AttributeTok{labels =} \FunctionTok{names}\NormalTok{(x), }\AttributeTok{col =} \FunctionTok{cm.colors}\NormalTok{(}\DecValTok{4}\NormalTok{), }\AttributeTok{main =} \StringTok{"3D Pie chart of Drivers, Van Drivers, Front and Rear seat passengers}\SpecialCharTok{\textbackslash{}n}\StringTok{killed or injured in Great Britain during  1969{-}1984 "}\NormalTok{, }\AttributeTok{labelcol =} \StringTok{"darkgreen"}\NormalTok{, }\AttributeTok{col.main =} \StringTok{"darkgreen"}\NormalTok{)}
\end{Highlighting}
\end{Shaded}

\includegraphics{B20102088-MuhammadHabibKhan-FinalLab_files/figure-latex/unnamed-chunk-3-1.pdf}

Above I have constructed a 3D pie chart using the plotrix library. The
pie chart shows the proportion of casualties in car accidents. An
average was taken of the classes through out the period to construct
this chart. The least are van drivers. This must be because vans are
usually less on the roads then sedan/coupe cars. Secondly, vans are
often driven slowly then the other cars. The most portion is taken by
the front seat passengers, followed by rear seat passengers and after
that comes the drivers at 3rd place. Surprisingly, according to the data
provided, passengers, front more than rear, are likely to meet the fate
of death than the one behind the wheel. One reason may be the fact that
most driver seats have airbags while the passengers seat do not.

\hypertarget{histogram}{%
\subsection{Histogram:}\label{histogram}}

\begin{Shaded}
\begin{Highlighting}[]
\CommentTok{\# Histogram for Drivers killed through out the period. Histogram is same as bar plot except that they depict the data in grouped form rather than discrete. }

\FunctionTok{hist}\NormalTok{(DriversKilled , }\AttributeTok{col=}\FunctionTok{terrain.colors}\NormalTok{(}\DecValTok{11}\NormalTok{) , }\AttributeTok{border =} \StringTok{\textquotesingle{}blue\textquotesingle{}}\NormalTok{, }\AttributeTok{main=}\StringTok{\textquotesingle{}Histogram For Number of Car Drivers Killed}\SpecialCharTok{\textbackslash{}n}\StringTok{during 1969{-}1984 in The Great Britain}\SpecialCharTok{\textbackslash{}n}\StringTok{ \textquotesingle{}}\NormalTok{, }\AttributeTok{xlab =} \StringTok{\textquotesingle{}Car Drivers Killed \textquotesingle{}}\NormalTok{ )}
\end{Highlighting}
\end{Shaded}

\includegraphics{B20102088-MuhammadHabibKhan-FinalLab_files/figure-latex/unnamed-chunk-4-1.pdf}

The histogram plotted demonstrates what was the usual toll of deaths of
car drivers in the given time period. As we can see, at most times the
casualties were between 100 and 120, peaking at 200 and hitting the
lowest at 60.

\hypertarget{scatter-plot}{%
\subsection{Scatter Plot:}\label{scatter-plot}}

\begin{Shaded}
\begin{Highlighting}[]
\CommentTok{\# Scatter Plot Between Increment in Petrol Prices \& Number of Drivers Killed through out the period }

\FunctionTok{plot}\NormalTok{(PetrolPrice, DriversKilled, }\AttributeTok{type=}\StringTok{\textquotesingle{}p\textquotesingle{}}\NormalTok{, }\AttributeTok{xlab =} \StringTok{\textquotesingle{}Petrol Price\textquotesingle{}}\NormalTok{, }\AttributeTok{ylab=}\StringTok{\textquotesingle{}Number of Car Drivers Killed\textquotesingle{}}\NormalTok{, }\AttributeTok{main=}\StringTok{\textquotesingle{}Petrol Prices VS Car Drivers Deaths\textquotesingle{}}\NormalTok{, }\AttributeTok{col=}\StringTok{\textquotesingle{}magenta\textquotesingle{}}\NormalTok{)}
\end{Highlighting}
\end{Shaded}

\includegraphics{B20102088-MuhammadHabibKhan-FinalLab_files/figure-latex/unnamed-chunk-5-1.pdf}

Above us we can see a scatter plot between increment in petrol prices
and number of drivers killed through out the period. If we try to look
at the relationship between the two classes, and try to predict a trend
based on this plot, we may find that the death of car drivers doesn't
depend on petrol prices much as they remained on average around the same
at each price point except for at the last one which is the highest
price which in logic should have more deaths but has the least. This may
have to do with the introduction of law of seat belts introduced and
being implemented by then and also the fact that car safety features
were becoming more and more common with airbags and other durability
features added to withstand such situations.

\hypertarget{box-plot}{%
\subsection{Box Plot:}\label{box-plot}}

\begin{Shaded}
\begin{Highlighting}[]
\CommentTok{\# Box plot of people killed/seriously injured during car accidents in Great Britain during 1969{-}1984. The box plots represents five values in the graph i.e., minimum, first quartile, second quartile (median), third quartile, the maximum value of the data vector.}

\FunctionTok{boxplot}\NormalTok{(drivers, }\AttributeTok{col=}\StringTok{\textquotesingle{}orange\textquotesingle{}}\NormalTok{, }\AttributeTok{xlab =} \StringTok{"Box Plot (Jan,1969 {-} Dec, 1984)"}\NormalTok{, }\AttributeTok{ylab =} \StringTok{"People killed/seriously injured during car accidents"}\NormalTok{, }\AttributeTok{col.axis =} \StringTok{"darkgreen"}\NormalTok{, }\AttributeTok{col.lab =} \StringTok{"dark blue"}\NormalTok{)}
\end{Highlighting}
\end{Shaded}

\includegraphics{B20102088-MuhammadHabibKhan-FinalLab_files/figure-latex/unnamed-chunk-6-1.pdf}

This box plot is based on the total amount of people killed or seriously
injured during the given period. This is the same as the data set
\emph{UKDriverDeaths}. There are five values that this plot tells us. As
we can see, the minimum amount of casualties was around 1000, then the
first quartile is a little less than 1500, the median (second quartile;
black line) is around 1700 while the third quartile is closing to 2000.
The maximum here can be seen close to 2500.

\hypertarget{correlation}{%
\section{Correlation}\label{correlation}}

\textbf{Definition:} When a variable tends to change from one to
another, whether direct or indirect, it is considered correlated. A
correlation coefficient is applied to measure a degree of association in
variables and is usually called Pearson's correlation coefficient. The
correlation coefficient is measured on a scale with values from +1
through 0 and -1. When both variables increase, the correlation is
positive, and if one variable increases, and the other decreases, the
correlation is negative.

\hypertarget{correlation-between-law-casualties}{%
\subsection{Correlation between Law \&
Casualties:}\label{correlation-between-law-casualties}}

Lets have a look at the most important correlation of the data set:

\begin{Shaded}
\begin{Highlighting}[]
\FunctionTok{plot}\NormalTok{(law, drivers, }\AttributeTok{pch =} \DecValTok{19}\NormalTok{, }\AttributeTok{col =} \StringTok{"pink"}\NormalTok{)}
\FunctionTok{abline}\NormalTok{(}\FunctionTok{lm}\NormalTok{(drivers }\SpecialCharTok{\textasciitilde{}}\NormalTok{ law), }\AttributeTok{col =} \StringTok{"red"}\NormalTok{, }\AttributeTok{lwd =} \DecValTok{3}\NormalTok{)}
\FunctionTok{text}\NormalTok{(}\FunctionTok{paste}\NormalTok{(}\StringTok{"Correlation:"}\NormalTok{, }\FunctionTok{round}\NormalTok{(}\FunctionTok{cor}\NormalTok{(law, drivers), }\DecValTok{2}\NormalTok{)), }\AttributeTok{x =} \FloatTok{0.4}\NormalTok{, }\AttributeTok{y =} \DecValTok{2300}\NormalTok{)}
\end{Highlighting}
\end{Shaded}

\includegraphics{B20102088-MuhammadHabibKhan-FinalLab_files/figure-latex/unnamed-chunk-7-1.pdf}
The above scatter plot depicts the correlation between the total amount
of people killed or seriously injured during the given period and if the
law was passed then or not (0 = not passed, 1 = passed). It is clearly
evident that the casualties decreases by a great deal after the law was
passed. The \emph{cor} commands tells us that the correlation
coefficient is negative 0.45 (-0.45) which proves that deaths decreases
as law is passed. The r-value in magnitude is 0.45 which tells that this
is a fairly moderate correlation. If we were to consider other factors
too, which includes the increased number of vehicles on the road or the
increased population, then this correlation is pretty good even without
those factors included. But to counter those extra factors, the safety
of the vehicles increased/improved too.

\hypertarget{correlation-of-the-entire-dataset}{%
\subsection{Correlation of the Entire
Dataset:}\label{correlation-of-the-entire-dataset}}

\begin{Shaded}
\begin{Highlighting}[]
\FunctionTok{library}\NormalTok{(corrplot)}
\end{Highlighting}
\end{Shaded}

\begin{verbatim}
## corrplot 0.92 loaded
\end{verbatim}

\begin{Shaded}
\begin{Highlighting}[]
\CommentTok{\# converting months into numeric data as cor command does not accept non{-}numeric data}
\NormalTok{Seatbelts}\SpecialCharTok{$}\NormalTok{Month }\OtherTok{\textless{}{-}} \FunctionTok{as.numeric}\NormalTok{(Month) }

\CommentTok{\# This command prints the correlation between all of the classes of the data set}
\FunctionTok{cor}\NormalTok{(Seatbelts)}
\end{Highlighting}
\end{Shaded}

\begin{verbatim}
##                     Year      Month DriversKilled    drivers      front
## Year           1.0000000 0.00000000    -0.3657291 -0.4809857 -0.6932829
## Month          0.0000000 1.00000000     0.5458411  0.5084645  0.4305932
## DriversKilled -0.3657291 0.54584115     1.0000000  0.8888264  0.7067596
## drivers       -0.4809857 0.50846452     0.8888264  1.0000000  0.8084114
## front         -0.6932829 0.43059322     0.7067596  0.8084114  1.0000000
## rear          -0.1969329 0.54996694     0.3533510  0.3436685  0.6202248
## kms            0.7739559 0.21251863    -0.3211016 -0.4447631 -0.3573823
## PetrolPrice    0.5109097 0.01030832    -0.3866061 -0.4576675 -0.5392394
## VanKilled     -0.5773166 0.14495270     0.4070412  0.4853995  0.4724207
## law            0.5619346 0.02555513    -0.3285051 -0.4452269 -0.5624455
##                      rear        kms PetrolPrice  VanKilled         law
## Year          -0.19693293  0.7739559  0.51090973 -0.5773166  0.56193462
## Month          0.54996694  0.2125186  0.01030832  0.1449527  0.02555513
## DriversKilled  0.35335102 -0.3211016 -0.38660609  0.4070412 -0.32850510
## drivers        0.34366850 -0.4447631 -0.45766754  0.4853995 -0.44522689
## front          0.62022476 -0.3573823 -0.53923937  0.4724207 -0.56244554
## rear           1.00000000  0.3330069 -0.13262721  0.1217581  0.02906753
## kms            0.33300689  1.0000000  0.38390038 -0.4980356  0.49054938
## PetrolPrice   -0.13262721  0.3839004  1.00000000 -0.2885584  0.39069323
## VanKilled      0.12175808 -0.4980356 -0.28855841  1.0000000 -0.39494121
## law            0.02906753  0.4905494  0.39069323 -0.3949412  1.00000000
\end{verbatim}

\begin{Shaded}
\begin{Highlighting}[]
\CommentTok{\# Converting into matrix to use with command *corrplot* }
\NormalTok{cor.mat.seatbelts }\OtherTok{=} \FunctionTok{cor}\NormalTok{(Seatbelts)}

\CommentTok{\# Plotting the correlation between all of the classes of the data set visually}
\FunctionTok{corrplot}\NormalTok{(cor.mat.seatbelts)}
\end{Highlighting}
\end{Shaded}

\includegraphics{B20102088-MuhammadHabibKhan-FinalLab_files/figure-latex/unnamed-chunk-8-1.pdf}

\begin{Shaded}
\begin{Highlighting}[]
\CommentTok{\# Another way of displaying correlation plot between all the classes, almost the same as corrplot command but this one displays the Pearson\textquotesingle{}s Coefficient written in numbers rather than fading colors}
\FunctionTok{library}\NormalTok{(psych)}
\end{Highlighting}
\end{Shaded}

\begin{verbatim}
## 
## Attaching package: 'psych'
\end{verbatim}

\begin{verbatim}
## The following object is masked from 'package:plotrix':
## 
##     rescale
\end{verbatim}

\begin{Shaded}
\begin{Highlighting}[]
\FunctionTok{corPlot}\NormalTok{(cor.mat.seatbelts, }\AttributeTok{cex =} \FloatTok{1.2}\NormalTok{)}
\end{Highlighting}
\end{Shaded}

\includegraphics{B20102088-MuhammadHabibKhan-FinalLab_files/figure-latex/unnamed-chunk-8-2.pdf}

The cor command prints the correlation between all of the classes of the
data set numerically in a tabular form. Using corrplot, we plot all that
data into a visually appealing and easy to grasp graph. Some main
selective insights from this correlation data:

\begin{itemize}
\item
  The petrol prices and years are moderately positively correlated,
  meaning that the petrol prices raised as the years passes
\item
  Years passed and law passed are weakly but close to moderately
  correlated with the total amount of people killed or seriously injured
  in car accidents. However, they both are moderately correlated with
  the deaths of front seat passengers. The correlation is negative in
  all instances showing that years passing resulted in less deaths.
\item
  Years passed and number of Van Drivers killed also has a moderate
  negative correlation.
\item
  The kms (distance) also has a weak but close to moderate negative
  correlation with the total amount of people killed or seriously
  injured in car accidents.
\end{itemize}

\end{document}
